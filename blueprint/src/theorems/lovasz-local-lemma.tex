\chapter{Lovász Local Lemma}

\begin{definition}[Dependency Graph]
    \label{def:Dependency_Graph}
    Let $\mathcal{A} = \{A_1, A_2, \dots, A_n\}$ be a finite set of events in a probability space. A graph $G=(V, E)$ is called a dependency graph for the events $\mathcal{A}$ if $V = \mathcal{A}$ and for each event $A_i \in \mathcal{A}$, the event $A_i$ is mutually independent of the set of all events $\{A_j \mid (A_i, A_j) \notin E\}$. For an event $A_i$, we denote its neighborhood in $G$ by $\Gamma(A_i)$.
\end{definition}

\begin{lemma}[Inequality for e]
    \label{lem:Inequality_for_e}
    For any positive integer $d \geq 1$, the following inequality holds:
    $$ \left(1 - \frac{1}{d+1}\right)^d \geq \frac{1}{e} $$
\end{lemma}

\begin{proof}
    A standard inequality for the mathematical constant $e$ states that for any real $x > 0$, $\left(1 + \frac{1}{x}\right)^x < e$. Let $x=d$. This gives $\left(1 + \frac{1}{d}\right)^d < e$.
    This is equivalent to $\left(\frac{d+1}{d}\right)^d < e$.
    Taking the reciprocal of both sides and reversing the inequality sign, we get $\left(\frac{d}{d+1}\right)^d > \frac{1}{e}$.
    Rewriting the base gives $\left(1 - \frac{1}{d+1}\right)^d > \frac{1}{e}$. The inequality in the lemma statement holds.
\end{proof}

\begin{lemma}[Conditional Probability Bound]
    \label{lem:Conditional_Probability_Bound}
    \uses{def:Dependency_Graph}
    Let $\mathcal{A} = \{A_1, \dots, A_n\}$ be a set of events with a dependency graph $G$. Suppose there exists an assignment of real numbers $x: \mathcal{A} \to [0,1)$ to the events such that for each $A \in \mathcal{A}$:
    $$ \Pr(A) \leq x(A) \prod_{B \in \Gamma(A)} (1-x(B)) $$
    Then for any event $A \in \mathcal{A}$ and any set $S \subseteq \mathcal{A} \setminus \{A\}$, we have:
    $$ \Pr\left(A \mid \bigwedge_{B \in S} \overline{B}\right) \leq x(A) $$
\end{lemma}

\begin{proof}
    We prove this by induction on the size of $S$, $k = |S|$.

    \textbf{Base Case ($k=0$):} If $S = \emptyset$, the statement is $\Pr(A) \leq x(A)$. From the lemma's hypothesis, we have $\Pr(A) \leq x(A) \prod_{B \in \Gamma(A)} (1-x(B))$. Since $x(B) \in [0,1)$ for all $B$, the product term is at most 1. Thus, $\Pr(A) \leq x(A)$, and the base case holds.

    \textbf{Inductive Step:} Assume the statement is true for all sets $S'$ with $|S'| < k$. Let $S$ be a set with $|S| = k > 0$. Let $S_1 = S \cap \Gamma(A)$ and $S_2 = S \setminus S_1$.
    Using the definition of conditional probability:
    $$ \Pr\left(A \mid \bigwedge_{B \in S} \overline{B}\right) = \frac{\Pr\left(A \wedge \bigwedge_{B \in S_1} \overline{B} \mid \bigwedge_{B \in S_2} \overline{B}\right)}{\Pr\left(\bigwedge_{B \in S_1} \overline{B} \mid \bigwedge_{B \in S_2} \overline{B}\right)} $$

    We bound the numerator:
    $$ \text{Numerator} \leq \Pr\left(A \mid \bigwedge_{B \in S_2} \overline{B}\right) $$
    By the definition of the dependency graph, $A$ is independent of all events in $S_2$. Thus,
    $$ \Pr\left(A \mid \bigwedge_{B \in S_2} \overline{B}\right) = \Pr(A) \leq x(A) \prod_{B \in \Gamma(A)} (1-x(B)) $$

    We bound the denominator. Let $S_1 = \{B_1, \dots, B_m\}$. Using the chain rule:
    $$ \Pr\left(\bigwedge_{j=1}^m \overline{B_j} \mid \bigwedge_{C \in S_2} \overline{C}\right) = \prod_{j=1}^m \Pr\left(\overline{B_j} \mid \bigwedge_{l=j+1}^m \overline{B_l} \wedge \bigwedge_{C \in S_2} \overline{C}\right) $$
    For each term in the product, the conditioning set has size $(m-j) + |S_2| < m + |S_2| = k$. By the induction hypothesis, $\Pr\left(B_j \mid \dots\right) \leq x(B_j)$. Thus, $\Pr\left(\overline{B_j} \mid \dots\right) \geq 1-x(B_j)$.
    So, the denominator is bounded below:
    $$ \text{Denominator} \geq \prod_{j=1}^m (1 - x(B_j)) = \prod_{B \in S_1} (1-x(B)) $$

    Combining the bounds for the numerator and denominator:
    $$ \Pr\left(A \mid \bigwedge_{B \in S} \overline{B}\right) \leq \frac{x(A) \prod_{B \in \Gamma(A)} (1-x(B))}{\prod_{B \in S_1} (1-x(B))} = x(A) \prod_{B \in \Gamma(A) \setminus S_1} (1-x(B)) $$
    Since $1-x(B) \leq 1$ for all $B$, the product is at most 1. Therefore, $\Pr\left(A \mid \bigwedge_{B \in S} \overline{B}\right) \leq x(A)$. The induction is complete.
\end{proof}

\begin{theorem}[Asymmetric Lovász Local Lemma]
    \label{thm:Asymmetric_Lovasz_Local_Lemma}
    \uses{def:Dependency_Graph}
    Let $\mathcal{A} = \{A_1, \ldots, A_n\}$ be a finite set of events with a dependency graph $G$. If there exists an assignment of real numbers $x: \mathcal{A} \to [0,1)$ such that for each $A \in \mathcal{A}$:
    $$ \Pr(A) \leq x(A) \prod_{B \in \Gamma(A)} (1-x(B)) $$
    Then the probability of avoiding all events in $\mathcal{A}$ is positive, and more specifically:
    $$ \Pr\left(\bigwedge_{i=1}^n \overline{A_i}\right) \geq \prod_{i=1}^n (1-x(A_i)) $$
\end{theorem}

\begin{proof}
    \uses{lem:Conditional_Probability_Bound}
    By the chain rule of probability, we have:
    $$ \Pr\left(\bigwedge_{i=1}^n \overline{A_i}\right) = \prod_{i=1}^n \Pr\left(\overline{A_i} \mid \bigwedge_{j=1}^{i-1} \overline{A_j}\right) $$
    Each term in this product can be written as:
    $$ \Pr\left(\overline{A_i} \mid \bigwedge_{j=1}^{i-1} \overline{A_j}\right) = 1 - \Pr\left(A_i \mid \bigwedge_{j=1}^{i-1} \overline{A_j}\right) $$
    By lemma \ref{lem:Conditional_Probability_Bound}, with the conditioning set $S = \{A_1, \dots, A_{i-1}\}$, we have:
    $$ \Pr\left(A_i \mid \bigwedge_{j=1}^{i-1} \overline{A_j}\right) \leq x(A_i) $$
    Therefore, each term in the product is bounded below:
    $$ \Pr\left(\overline{A_i} \mid \bigwedge_{j=1}^{i-1} \overline{A_j}\right) \geq 1 - x(A_i) $$
    Substituting this back into the chain rule expansion gives:
    $$ \Pr\left(\bigwedge_{i=1}^n \overline{A_i}\right) \geq \prod_{i=1}^n (1 - x(A_i)) $$
    Since $x(A_i) \in [0,1)$ for all $i$, each term $(1-x(A_i))$ is positive, and thus the product is strictly positive.
\end{proof}

\begin{theorem}[Symmetric Lovász Local Lemma]
    \label{thm:Symmetric_Lovasz_Local_Lemma}
    \uses{def:Dependency_Graph}
    Let $\mathcal{A} = \{A_1, \ldots, A_n\}$ be a set of events. Suppose that for each $i$, $\Pr(A_i) \leq p$, and each event $A_i$ is dependent on at most $d$ other events from $\mathcal{A}$. If
    $$ ep(d+1) \leq 1 $$
    where $e$ is the base of the natural logarithm, then $\Pr(\bigwedge_{i=1}^n \overline{A_i}) > 0$.
\end{theorem}

\begin{proof}
    \uses{thm:Asymmetric_Lovasz_Local_Lemma}
    \uses{lem:Inequality_for_e}
    We aim to satisfy the conditions of the theorem \ref{thm:Asymmetric_Lovasz_Local_Lemma}. Let the dependency graph be such that the maximum degree is at most $d$. We define $x(A_i) = \frac{1}{d+1}$ for all $A_i \in \mathcal{A}$. Since $d \geq 0$, $x(A_i) \in (0, 1]$. For the condition $x(A_i) < 1$ to hold, we require $d>0$. If $d=0$, events are mutually independent and the problem is trivial. Let's assume $d>0$.

    We must verify the condition from theorem \ref{thm:Asymmetric_Lovasz_Local_Lemma}:
    $$ \Pr(A_i) \leq x(A_i) \prod_{A_j \in \Gamma(A_i)} (1-x(A_j)) $$
    Given $\Pr(A_i) \leq p$ and $|\Gamma(A_i)| \leq d$, it is sufficient to show:
    $$ p \leq \frac{1}{d+1} \left(1-\frac{1}{d+1}\right)^{|\Gamma(A_i)|} $$
    Since $1 - \frac{1}{d+1} < 1$, the term on the right is minimized when the exponent is maximal. Thus, it suffices to show:
    $$ p \leq \frac{1}{d+1} \left(1-\frac{1}{d+1}\right)^d $$
    By lemma \ref{lem:Inequality_for_e}, we have $\left(1-\frac{1}{d+1}\right)^d \geq \frac{1}{e}$.
    Therefore, the condition is satisfied if we have:
    $$ p \leq \frac{1}{d+1} \cdot \frac{1}{e} $$
    This is equivalent to the given condition $ep(d+1) \leq 1$.
    Since the conditions of theorem \ref{thm:Asymmetric_Lovasz_Local_Lemma} are met, we can conclude that $\Pr(\bigwedge_{i=1}^n \overline{A_i}) > 0$.
\end{proof}
