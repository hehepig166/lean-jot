\chapter{Lovász Local Lemma}

\begin{definition}[Dependency Graph]
Let $\mathcal{A} = \{A_1, A_2, \ldots, A_n\}$ be a finite set of events in a probability space. A graph $D=(V,E)$ with $V=\{1, 2, \ldots, n\}$ is called a **dependency graph** for the events $\mathcal{A}$ if for each event $A_i$, it is mutually independent of the set of events $\{A_j \mid (i,j) \notin E \text{ and } i \neq j\}$. The neighborhood of an event $A_i$ is denoted by $\Gamma(A_i) = \{A_j \mid (i,j) \in E\}$.
\end{definition}

\begin{lemma}[Conditional Probability Bound]
Let $\mathcal{A} = \{A_1, \ldots, A_n\}$ be a finite set of events with a dependency graph $D$. If there exists an assignment of real numbers $x: \mathcal{A} \to [0, 1)$ such that for all $A \in \mathcal{A}$,
$$\Pr(A) \leq x(A) \prod_{B \in \Gamma(A)} (1 - x(B))$$
then for any event $A_i \in \mathcal{A}$ and any set $S \subseteq \mathcal{A} \setminus \{A_i\}$, we have
$$\Pr\left(A_i \mid \bigwedge_{B \in S} \overline{B}\right) \leq x(A_i)$$
\end{lemma}

\begin{proof}
We proceed by induction on the size of the set $S$.

\textbf{Base case ($|S|=0$):} Let $S = \emptyset$. The condition is $\Pr(A_i) \leq x(A_i)$. From the lemma's hypothesis, we have $\Pr(A_i) \leq x(A_i) \prod_{B \in \Gamma(A_i)} (1 - x(B))$. Since $x(B) \in [0, 1)$, the product term is at most 1, so $\Pr(A_i) \leq x(A_i)$. The base case holds.

\textbf{Inductive step:} Assume the statement holds for all sets $S'$ with $|S'| < k$. Let $S$ be a set with $|S| = k$. Let $A_i \in \mathcal{A} \setminus S$. We want to show $\Pr(A_i \mid \bigwedge_{B \in S} \overline{B}) \leq x(A_i)$.

Let $S_1 = S \cap \Gamma(A_i)$ be the neighbors of $A_i$ in $S$, and let $S_2 = S \setminus S_1$. By the definition of conditional probability,
$$\Pr\left(A_i \mid \bigwedge_{B \in S} \overline{B}\right) = \frac{\Pr\left(A_i \wedge \bigwedge_{B \in S_1} \overline{B} \mid \bigwedge_{C \in S_2} \overline{C}\right)}{\Pr\left(\bigwedge_{B \in S_1} \overline{B} \mid \bigwedge_{C \in S_2} \overline{C}\right)}$$
We bound the numerator and the denominator separately.

For the numerator, we have:
$$\Pr\left(A_i \wedge \bigwedge_{B \in S_1} \overline{B} \mid \bigwedge_{C \in S_2} \overline{C}\right) \leq \Pr\left(A_i \mid \bigwedge_{C \in S_2} \overline{C}\right)$$
Since $A_i$ is independent of the events in $S_2$ by the definition of the dependency graph, this simplifies to $\Pr(A_i)$. By the hypothesis of the lemma, we have:
$$\text{Numerator} \leq \Pr(A_i) \leq x(A_i) \prod_{B \in \Gamma(A_i)} (1 - x(B))$$
For the denominator, let $S_1 = \{B_1, \ldots, B_m\}$. We expand using the chain rule:
$$\Pr\left(\bigwedge_{j=1}^m \overline{B_j} \mid \bigwedge_{C \in S_2} \overline{C}\right) = \prod_{j=1}^m \Pr\left(\overline{B_j} \mid \bigwedge_{l=j+1}^m \overline{B_l} \wedge \bigwedge_{C \in S_2} \overline{C}\right)$$
For each term in the product, we have:
$$\Pr\left(\overline{B_j} \mid \dots\right) = 1 - \Pr\left(B_j \mid \dots\right)$$
The conditioning set for $B_j$ is $\{B_{j+1}, \ldots, B_m\} \cup S_2$, which is a subset of $\mathcal{A} \setminus \{B_j\}$ and has size strictly less than $k=|S|$. By the induction hypothesis, $\Pr(B_j \mid \dots) \leq x(B_j)$. Therefore, each term is $\geq (1 - x(B_j))$. This gives us the bound:
$$\text{Denominator} \geq \prod_{j=1}^m (1 - x(B_j)) = \prod_{B \in S_1} (1 - x(B))$$
Combining the bounds for the numerator and denominator:
$$\Pr\left(A_i \mid \bigwedge_{B \in S} \overline{B}\right) \leq \frac{x(A_i) \prod_{B \in \Gamma(A_i)} (1 - x(B))}{\prod_{B \in S_1} (1 - x(B))} = x(A_i) \prod_{B \in \Gamma(A_i) \setminus S_1} (1 - x(B))$$
Since $x(B) \in [0,1)$, the product is at most 1. Thus, we have $\Pr(A_i \mid \bigwedge_{B \in S} \overline{B}) \leq x(A_i)$. The induction is complete.
\end{proof}

\begin{theorem}[Asymmetric Lovász Local Lemma]
Let $\mathcal{A}=\{A_{1},\ldots ,A_{n}\}$ be a finite set of events with a dependency graph $D$. If there exists an assignment of real numbers $x: \mathcal{A} \to [0,1)$ such that for each $A_i \in \mathcal{A}$,
$$\Pr(A_i) \leq x(A_i)\prod_{A_j \in \Gamma(A_i)}(1-x(A_j))$$
then the probability that none of the events in $\mathcal{A}$ occurs is strictly positive. Specifically,
$$\Pr\left(\bigwedge_{i=1}^n \overline{A_{i}}\right) \geq \prod_{i=1}^n (1-x(A_{i})) > 0$$
\end{theorem}

\begin{proof}
Using the chain rule for probability, we can write:
$$\Pr\left(\bigwedge_{i=1}^n \overline{A_i}\right) = \Pr(\overline{A_1}) \cdot \Pr(\overline{A_2} \mid \overline{A_1}) \cdots \Pr\left(\overline{A_n} \mid \bigwedge_{i=1}^{n-1} \overline{A_i}\right) = \prod_{i=1}^n \Pr\left(\overline{A_i} \mid \bigwedge_{j=1}^{i-1} \overline{A_j}\right)$$
Each term in the product can be rewritten as:
$$\Pr\left(\overline{A_i} \mid \bigwedge_{j=1}^{i-1} \overline{A_j}\right) = 1 - \Pr\left(A_i \mid \bigwedge_{j=1}^{i-1} \overline{A_j}\right)$$
By the Conditional Probability Bound Lemma, with $S = \{A_1, \ldots, A_{i-1}\}$, we have $\Pr\left(A_i \mid \bigwedge_{j=1}^{i-1} \overline{A_j}\right) \leq x(A_i)$.
Therefore, each term in the product is bounded below:
$$\Pr\left(\overline{A_i} \mid \bigwedge_{j=1}^{i-1} \overline{A_j}\right) \geq 1 - x(A_i)$$
Combining these inequalities, we get the final result:
$$\Pr\left(\bigwedge_{i=1}^n \overline{A_i}\right) \geq \prod_{i=1}^n (1 - x(A_i))$$
Since $x(A_i) < 1$ for all $i$, the product on the right is strictly positive.
\end{proof}

\begin{theorem}[Symmetric Lovász Local Lemma]
Let $A_1, A_2, \ldots, A_k$ be a sequence of events. Suppose that for some real numbers $p$ and $d$:
\begin{enumerate}
    \item $\Pr(A_i) \leq p$ for all $i$.
    \item Each event $A_i$ is dependent on at most $d$ other events.
\end{enumerate}
If $ep(d+1) \leq 1$, where $e$ is Euler's number, then there is a nonzero probability that none of the events occurs, i.e., $\Pr(\bigwedge_{i=1}^k \overline{A_i}) > 0$.
\end{theorem}

\begin{proof}
This is a corollary of the Asymmetric Lovász Local Lemma. We define $x(A_i) = \frac{1}{d+1}$ for all $i$. Since $d \ge 0$, we have $x(A_i) \in (0, 1]$. We need to check if the condition of the asymmetric lemma holds:
$$\Pr(A_i) \leq \frac{1}{d+1} \prod_{A_j \in \Gamma(A_i)} \left(1 - \frac{1}{d+1}\right)$$
The left side is at most $p$. The number of neighbors $|\Gamma(A_i)|$ is at most $d$. Thus, the product on the right is at least:
$$\frac{1}{d+1} \left(1 - \frac{1}{d+1}\right)^d$$
We use the well-known inequality $(1 - 1/t)^{t-1} > 1/e$ for $t > 1$. Let $t=d+1$. Then $\left(1 - \frac{1}{d+1}\right)^d > 1/e$.
So, the condition becomes:
$$p \leq \frac{1}{d+1} \left(1 - \frac{1}{d+1}\right)^d > \frac{1}{e(d+1)}$$
The given hypothesis is $ep(d+1) \leq 1$, which is equivalent to $p \leq \frac{1}{e(d+1)}$. This is a stricter condition than what is required to apply the Asymmetric Lovász Local Lemma. Therefore, the conditions are met.

By the Asymmetric Lovász Local Lemma,
$$\Pr\left(\bigwedge_{i=1}^k \overline{A_i}\right) \geq \prod_{i=1}^k \left(1 - \frac{1}{d+1}\right) = \left(\frac{d}{d+1}\right)^k > 0$$
Thus, there is a positive probability that none of the events occurs.
\end{proof}
