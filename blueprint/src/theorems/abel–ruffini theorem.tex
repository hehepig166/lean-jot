\chapter{Abel–Ruffini Theorem}

\begin{definition}[Solvability by Radicals]
A polynomial equation $p(x) = 0$ with coefficients in a field $F_0$ is said to be \textbf{solvable by radicals} if all its roots can be expressed using only the elements of $F_0$ and the operations of addition, subtraction, multiplication, division, and $n$-th root extraction.
This is equivalent to the existence of a tower of field extensions
$$F_0 \subseteq F_1 \subseteq \dots \subseteq F_k$$
such that the splitting field of $p(x)$ is a subfield of $F_k$, and for each $i \in \{1, \dots, k\}$, the extension $F_i$ is a radical extension of $F_{i-1}$, meaning $F_i = F_{i-1}(\alpha_i)$ where $\alpha_i^{n_i} \in F_{i-1}$ for some integer $n_i > 1$.
\end{definition}



\begin{lemma}[Solvability Condition in Galois Theory]
A polynomial equation is solvable by radicals if and only if its Galois group is a solvable group.
\end{lemma}

\begin{proof}
Let $p(x)$ be a polynomial over a field $F_0$, and let $L$ be its splitting field. The equation $p(x)=0$ is solvable by radicals if and only if there exists a tower of fields $F_0 \subseteq F_1 \subseteq \dots \subseteq F_k$ such that $L \subseteq F_k$ and each extension $F_i/F_{i-1}$ is a radical extension.

We can refine this tower by adjoining roots of unity at each step, ensuring each extension is a normal extension. A radical extension $F_{i-1}(\alpha_i)$ where $\alpha_i^{n_i} \in F_{i-1}$ can be embedded in a normal extension by first adjoining a primitive $n_i$-th root of unity. The resulting Galois group for each step, $\text{Gal}(F_i/F_{i-1})$, is cyclic, and therefore abelian.

By the Galois correspondence, this tower of fields corresponds to a sequence of subgroups of the Galois group $G = \text{Gal}(L/F_0)$:
$$G = G_0 \triangleright G_1 \triangleright \dots \triangleright G_k = \{e\}$$
where each $G_i$ is a normal subgroup of $G_{i-1}$, and the quotient group $G_{i-1}/G_i$ is isomorphic to $\text{Gal}(F_i/F_{i-1})$. Since each of these quotient groups is abelian, this composition series has abelian factors. This is precisely the definition of a \textbf{solvable group}.

Conversely, if the Galois group $G = \text{Gal}(L/F_0)$ is solvable, it has a composition series with abelian (and thus cyclic, as every finite abelian group is a product of cyclic groups) factors. By the Galois correspondence, this corresponds to a tower of normal field extensions where each step is a cyclic extension. A cyclic extension of degree $n$ over a field containing the $n$-th roots of unity is a radical extension. Therefore, the roots of $p(x)$ can be expressed by radicals.
\end{proof}



\begin{lemma}[Insolvability of the Symmetric Group for $n \ge 5$]
The symmetric group $S_n$ is not a solvable group for any integer $n \ge 5$.
\end{lemma}

\begin{proof}
A group $G$ is solvable if it has a composition series
$$G = G_0 \triangleright G_1 \triangleright \dots \triangleright G_k = \{e\}$$
where each $G_i$ is a normal subgroup of $G_{i-1}$ and each factor group $G_{i-1}/G_i$ is abelian.

Consider the symmetric group $S_n$ for $n \ge 5$. The only non-trivial proper normal subgroup of $S_n$ is the alternating group, $A_n$. Thus, any composition series for $S_n$ must begin with $S_n \triangleright A_n$. The quotient group is $S_n/A_n \cong \mathbb{Z}_2$, which is abelian.

The next step requires finding a non-trivial proper normal subgroup of $A_n$. However, for $n \ge 5$, the alternating group $A_n$ is a \textbf{simple group}, which means it has no non-trivial proper normal subgroups.
Therefore, the only possible composition series is:
$$S_n \triangleright A_n \triangleright \{e\}$$
The factor groups are $S_n/A_n \cong \mathbb{Z}_2$ (which is abelian) and $A_n/\{e\} \cong A_n$.

For $A_n$ to be solvable, it must be abelian itself (since it cannot be broken down further). However, the group $A_n$ is not abelian for $n \ge 4$. For example, in $A_5$, the permutations $(1 \ 2 \ 3)$ and $(3 \ 4 \ 5)$ are both in $A_5$, but $(1 \ 2 \ 3)(3 \ 4 \ 5) = (1 \ 2 \ 3 \ 4 \ 5)$ whereas $(3 \ 4 \ 5)(1 \ 2 \ 3) = (1 \ 2 \ 4 \ 5 \ 3)$. These are not equal.

Since the factor group $A_n$ in the composition series is not abelian, the group $S_n$ is not solvable for $n \ge 5$.
\end{proof}



\begin{lemma}[Galois Group of the General Polynomial]
For any integer $n > 0$, the Galois group of the general polynomial equation of degree $n$ with indeterminate coefficients over the field of rational numbers $\mathbb{Q}$ is the symmetric group $S_n$.
\end{lemma}

\begin{proof}
Let the general polynomial be
$$P(x) = x^n + a_1 x^{n-1} + \dots + a_n = 0$$
where the coefficients $a_1, \dots, a_n$ are algebraically independent indeterminates over $\mathbb{Q}$. Let the roots of this polynomial be $x_1, \dots, x_n$. The base field is $K = \mathbb{Q}(a_1, \dots, a_n)$. The splitting field is $L = \mathbb{Q}(x_1, \dots, x_n)$.

By Vieta's formulas, the coefficients $a_i$ are the elementary symmetric polynomials in the roots $x_j$:
\begin{align*}
a_1 &= -(x_1 + x_2 + \dots + x_n) \\
a_2 &= x_1x_2 + x_1x_3 + \dots + x_{n-1}x_n \\
&\vdots \\
a_n &= (-1)^n x_1x_2\cdots x_n
\end{align*}
The Galois group $\text{Gal}(L/K)$ consists of automorphisms of $L$ that fix every element of $K$. Since the coefficients $a_i$ are in $K$, any automorphism $\sigma \in \text{Gal}(L/K)$ must leave the coefficients unchanged. An automorphism of the splitting field $L$ is completely determined by its action on the roots $x_1, \dots, x_n$, and it must permute these roots. Thus, $\text{Gal}(L/K)$ is a subgroup of the symmetric group $S_n$.

Because the coefficients $a_i$ are the elementary symmetric polynomials, they are invariant under *any* permutation of the roots $x_j$. This means that every permutation of the roots gives rise to a valid automorphism of $L$ that fixes $K$. Therefore, the Galois group is the entire symmetric group, $\text{Gal}(L/K) = S_n$.
\end{proof}



\begin{theorem}[Abel–Ruffini Theorem]
The general polynomial equation of degree $n$ cannot be solved by radicals for $n \ge 5$.
\end{theorem}

\begin{proof}
Let the general polynomial equation of degree $n$ be given.
\begin{enumerate}
    \item By Lemma 3, the Galois group of this equation is the symmetric group $S_n$.
    \item By Lemma 2, for $n \ge 5$, the group $S_n$ is not solvable.
    \item By Lemma 1, a polynomial equation can be solved by radicals if and only if its Galois group is solvable.
\end{enumerate}
Therefore, the general polynomial equation of degree $n \ge 5$ is not solvable by radicals.
\end{proof}



\begin{theorem}[Existence of an Unsolvable Quintic]
There exists a polynomial equation of degree five with rational coefficients that is not solvable by radicals.
\end{theorem}

\begin{proof}
Consider the polynomial $p(x) = x^5 - x - 1$ over the rational numbers $\mathbb{Q}$.
\begin{enumerate}
    \item The polynomial is irreducible over $\mathbb{Q}$.
    \item The derivative is $p'(x) = 5x^4 - 1$, which has two real roots $\pm 1/\sqrt[4]{5}$. This means $p(x)$ has two critical points, and thus at most three real roots. A calculation of values ($p(-2) = -31$, $p(-1)= -1$, $p(0)=-1$, $p(1)=-1$, $p(2)=29$) shows it has exactly three real roots and therefore two complex conjugate roots. Complex conjugation is an automorphism of the splitting field that fixes the real roots and swaps the two complex roots. This automorphism corresponds to a transposition in the Galois group $G$.
    \item Modulo 2, the polynomial factors into $(x^2+x+1)(x^3+x^2+1)$, which are irreducible over $\mathbb{F}_2$. By a theorem of Dedekind, this implies that the Galois group $G$ contains a permutation with cycle structure $(2,3)$. The cube of this permutation is a transposition.
    \item Modulo 3, the polynomial $x^5 - x - 1$ is irreducible over $\mathbb{F}_3$. This implies that the Galois group $G$ contains a 5-cycle.
    \item A subgroup of $S_5$ that contains both a transposition and a 5-cycle must be the entire group $S_5$. Therefore, the Galois group of $x^5 - x - 1$ is $S_5$.
\end{enumerate}
From Lemma 2, since $S_5$ is not solvable, the equation $x^5 - x - 1 = 0$ is not solvable by radicals.
\end{proof}
