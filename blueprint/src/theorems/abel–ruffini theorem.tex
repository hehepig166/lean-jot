\chapter{Abel-Ruffini Theorem}

\begin{definition}[Solution by Radicals]
    \label{def:Solution_by_Radicals}
    A field extension $L/F$ is a \textit{radical extension} if there exists a tower of fields $F = F_0 \subseteq F_1 \subseteq \dots \subseteq F_k = L$ such that for each $i \in \{1, \dots, k\}$, we have $F_i = F_{i-1}(\alpha_i)$ for some $\alpha_i \in F_i$ where $\alpha_i^{n_i} \in F_{i-1}$ for some integer $n_i > 0$.

    A polynomial $p(x) \in F[x]$ is \textit{solvable by radicals} if there exists a radical extension $L/F$ that contains the splitting field of $p(x)$. This means that all roots of $p(x)$ can be expressed using only the coefficients of $p(x)$, the field operations (addition, subtraction, multiplication, division), and root extractions.
\end{definition}

\begin{definition}[Solvable Group]
    \label{def:Solvable_Group}
    A group $G$ is called \textit{solvable} if there exists a finite chain of subgroups
    \[ \{e\} = G_0 \trianglelefteq G_1 \trianglelefteq \dots \trianglelefteq G_k = G \]
    such that for each $i \in \{1, \dots, k\}$, the quotient group $G_i / G_{i-1}$ is abelian. Such a chain is called a subnormal series with abelian factors.
\end{definition}

\begin{lemma}[Galois's Criterion for Solvability]
    \label{lem:Solvability_Criterion}
    \uses{def:Solution_by_Radicals}
    \uses{def:Solvable_Group}
    A polynomial $p(x) \in F[x]$ (where $F$ is a field of characteristic 0) is solvable by radicals if and only if its Galois group, $\operatorname{Gal}(K/F)$, is a solvable group, where $K$ is the splitting field of $p(x)$ over $F$.
\end{lemma}

\begin{proof}
    This is a fundamental theorem of Galois theory. The proof connects the structure of a radical field extension to the group-theoretic property of solvability.

    ($\Rightarrow$) If $p(x)$ is solvable by radicals, its splitting field $K$ is contained in a radical extension $L/F$. We can construct a tower of fields $F=F_0 \subset F_1 \subset \dots \subset F_m=L$ where each step is a simple radical extension. By adjoining roots of unity, we can ensure each extension $F_i/F_{i-1}$ is a normal extension with an abelian (in fact, cyclic) Galois group. The Galois group $\operatorname{Gal}(L/F)$ is then solvable. Since $K \subseteq L$, $\operatorname{Gal}(K/F)$ is a quotient group of $\operatorname{Gal}(L/F)$, and a quotient of a solvable group is solvable.

    ($\Leftarrow$) If $\operatorname{Gal}(K/F)$ is solvable, there exists a composition series with abelian quotients. By the Galois correspondence, this corresponds to a tower of intermediate fields $F=K_0 \subset K_1 \subset \dots \subset K_m=K$ where each extension $K_i/K_{i-1}$ is a normal extension with an abelian Galois group. By adjoining roots of unity and applying Kummer theory, each of these abelian extensions can be shown to be a radical extension. Thus, $K$ is contained in a radical extension of $F$, so $p(x)$ is solvable by radicals.
\end{proof}

\begin{lemma}[Insolvability of the Alternating Group $A_n$ for $n \geq 5$]
    \label{lem:Insolvability_of_An}
    \uses{def:Solvable_Group}
    For any integer $n \geq 5$, the alternating group $A_n$ is not a solvable group.
\end{lemma}

\begin{proof}
    A standard result in group theory states that for $n \geq 5$, the alternating group $A_n$ is a simple group. A group is simple if its only normal subgroups are the trivial group $\{e\}$ and the group itself.
    For $A_n$ to be solvable, as per definition \ref{def:Solvable_Group}, there must be a subnormal series with abelian factors. If $A_n$ were solvable, its composition series would be $\{e\} \trianglelefteq A_n$. The corresponding quotient group would be $A_n / \{e\} \cong A_n$.
    However, for $n \geq 5$, the group $A_n$ is non-abelian. For example, in $A_5$, the permutations $(1 \ 2 \ 3)$ and $(3 \ 4 \ 5)$ do not commute:
    \[ (1 \ 2 \ 3)(3 \ 4 \ 5) = (1 \ 2 \ 4 \ 5 \ 3) \]
    \[ (3 \ 4 \ 5)(1 \ 2 \ 3) = (1 \ 2 \ 3 \ 4 \ 5) \]
    Since the only quotient in the composition series, $A_n$ itself, is not abelian, $A_n$ cannot be a solvable group.
\end{proof}

\begin{lemma}[Insolvability of the Symmetric Group $S_n$ for $n \geq 5$]
    \label{lem:Insolvability_of_Sn}
    \uses{lem:Insolvability_of_An}
    For any integer $n \geq 5$, the symmetric group $S_n$ is not a solvable group.
\end{lemma}

\begin{proof}
    Consider the subnormal series for $S_n$: $\{e\} \trianglelefteq A_n \trianglelefteq S_n$. The alternating group $A_n$ is a normal subgroup of $S_n$. The quotient group $S_n / A_n$ is isomorphic to the cyclic group of order 2, which is abelian.
    However, for $S_n$ to be solvable, every subgroup in the composition series must be solvable. The subgroup $A_n$ is not solvable for $n \geq 5$ by lemma \ref{lem:Insolvability_of_An}. Since a subgroup of a solvable group must be solvable, $S_n$ cannot be solvable if it contains a non-solvable subgroup.
    Therefore, for $n \geq 5$, the symmetric group $S_n$ is not solvable.
\end{proof}

\begin{lemma}[Galois Group of the General Polynomial]
    \label{lem:Galois_Group_of_General_Polynomial}
    The Galois group of the general polynomial equation of degree $n$,
    \[ x^n + a_1 x^{n-1} + \dots + a_{n-1}x + a_n = 0 \]
    where the coefficients $a_1, \dots, a_n$ are algebraically independent indeterminates over $\mathbb{Q}$, is the symmetric group $S_n$.
\end{lemma}

\begin{proof}
    Let $F = \mathbb{Q}(a_1, \dots, a_n)$ be the field of coefficients. Let $x_1, \dots, x_n$ be the roots of the polynomial. The coefficients $a_i$ are the elementary symmetric polynomials in the roots $x_i$ (up to a sign). The splitting field of the polynomial is $K = F(x_1, \dots, x_n)$. Since the $a_i$ are symmetric in the $x_i$, any permutation of the roots $x_i$ leaves the coefficients $a_i$ unchanged, and thus leaves the base field $F$ fixed.

    The group of automorphisms of $K$ that permutes the roots is therefore the full symmetric group $S_n$. Since the $x_i$ are indeterminates over $\mathbb{Q}$, there are no algebraic relations among them other than those imposed by the symmetric polynomials. Therefore, every permutation of the roots corresponds to a valid automorphism in the Galois group over the field $F$ generated by the symmetric polynomials. Thus, $\operatorname{Gal}(K/F) \cong S_n$.
\end{proof}

\begin{theorem}[Abel-Ruffini Theorem, General Form]
    \label{thm:General_Abel-Ruffini}
    \uses{lem:Galois_Group_of_General_Polynomial}
    \uses{lem:Insolvability_of_Sn}
    \uses{lem:Solvability_Criterion}
    The general polynomial equation of degree $n \geq 5$ is not solvable by radicals.
\end{theorem}

\begin{proof}
    By lemma \ref{lem:Galois_Group_of_General_Polynomial}, the Galois group of the general polynomial equation of degree $n$ over the field of its coefficients is the symmetric group $S_n$.
    By lemma \ref{lem:Insolvability_of_Sn}, the group $S_n$ is not solvable for $n \geq 5$.
    By lemma \ref{lem:Solvability_Criterion}, a polynomial is solvable by radicals if and only if its Galois group is solvable.
    Combining these results, the general polynomial equation of degree $n \geq 5$ is not solvable by radicals.
\end{proof}

\begin{lemma}[A Quintic with Galois Group $S_5$]
    \label{lem:Quintic_with_S5_Galois_Group}
    The polynomial $p(x) = x^5 - x - 1$ has the symmetric group $S_5$ as its Galois group over the field of rational numbers $\mathbb{Q}$.
\end{lemma}

\begin{proof}
    Let $G = \operatorname{Gal}(p(x)/\mathbb{Q})$ be the Galois group of $p(x)$. We identify $G$ as a subgroup of $S_5$.

    \begin{enumerate}
        \item \textbf{Irreducibility}: The polynomial $p(x)$ is irreducible over $\mathbb{Q}$. We check this by reducing modulo 3. Over $\mathbb{F}_3$, $p(x) = x^5 - x - 1$ has no roots, as $p(0) = -1$, $p(1) = -1$, and $p(-1) = -1$. If it were reducible, it would have an irreducible factor of degree 1 or 2. Since it has no roots, it has no linear factors. The only irreducible quadratic over $\mathbb{F}_3$ is $x^2+1$. Division gives $x^5 - x - 1 = (x^2+1)(x^3-x) - 1$, so $x^2+1$ is not a factor. Thus, $p(x)$ is irreducible over $\mathbb{F}_3$, which implies it is irreducible over $\mathbb{Q}$. Since $p(x)$ is irreducible, its Galois group $G$ is a transitive subgroup of $S_5$.

        \item \textbf{Existence of a 5-cycle}: Since $p(x)$ is irreducible of degree 5 over $\mathbb{F}_3$, Dedekind's theorem implies that $G$ contains a permutation of cycle length 5 (a 5-cycle).

        \item \textbf{Existence of a transposition}: We reduce $p(x)$ modulo 2. Over $\mathbb{F}_2$, we have $p(x) = x^5 - x - 1 = x^5 + x + 1$. We can check that this polynomial factors as $(x^2+x+1)(x^3+x^2+1)$. Both factors are irreducible over $\mathbb{F}_2$. By Dedekind's theorem, the Galois group $G$ contains a permutation $\sigma$ with a cycle structure of $(2,3)$, i.e., a product of a disjoint 2-cycle and a 3-cycle.

        \item \textbf{Generation of $S_5$}: If we take the permutation $\sigma$ of cycle structure $(2,3)$, then $\sigma^3$ is a transposition (a 2-cycle). We have shown that $G$ contains a 5-cycle and a transposition. A standard result in group theory states that any subgroup of $S_p$ (for $p$ prime) that contains a $p$-cycle and a transposition must be the entire group $S_p$. Here, $p=5$.
    \end{enumerate}
    Therefore, the Galois group $G$ is isomorphic to $S_5$.
\end{proof}

\begin{theorem}[Abel-Ruffini Theorem, Specific Form]
    \label{thm:Specific_Abel-Ruffini}
    \uses{lem:Quintic_with_S5_Galois_Group}
    \uses{lem:Insolvability_of_Sn}
    \uses{lem:Solvability_Criterion}
    There exists a polynomial equation of degree five with rational coefficients that is not solvable by radicals.
\end{theorem}

\begin{proof}
    Consider the polynomial equation $x^5 - x - 1 = 0$.
    By lemma \ref{lem:Quintic_with_S5_Galois_Group}, the Galois group of this polynomial over $\mathbb{Q}$ is the symmetric group $S_5$.
    By lemma \ref{lem:Insolvability_of_Sn}, the group $S_5$ is not solvable.
    By lemma \ref{lem:Solvability_Criterion}, a polynomial is solvable by radicals if and only if its Galois group is solvable.
    Therefore, the equation $x^5 - x - 1 = 0$ is not solvable by radicals.
\end{proof}
