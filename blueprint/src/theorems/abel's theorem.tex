
\chapter{Abel's Theorem}

\begin{definition}[Stolz Sector]
    \label{def:Stolz_Sector}
    For a fixed finite constant $M > 1$, a Stolz sector is a region within the open unit disk $\{z \in \mathbb{C} : |z| < 1\}$ defined by the inequality
    \[
        |1-z| \leq M(1-|z|).
    \]
    This region is a curvilinear triangle with a vertex at $z=1$, tangent to the unit circle. A point $z$ approaches $1$ within a Stolz sector if it approaches $1$ while remaining within such a region.
\end{definition}

\begin{lemma}[Summation by Parts for Power Series]
    \label{lem:Summation_by_Parts}
    Let $G(z) = \sum_{k=0}^{\infty} a_k z^k$ be a power series with radius of convergence at least 1. Let $s_n = \sum_{k=0}^{n} a_k$ be the partial sums of its coefficients, with $s_{-1}=0$. Then for any $z$ in the open unit disk ($|z| < 1$), the series for $G(z)$ can be expressed as:
    \[
        G(z) = (1-z) \sum_{k=0}^{\infty} s_k z^k.
    \]
\end{lemma}

\begin{proof}
    We begin by expressing the coefficients $a_k$ in terms of the partial sums: $a_k = s_k - s_{k-1}$ for $k \geq 0$.
    Consider the partial sum of the power series, $G_N(z) = \sum_{k=0}^{N} a_k z^k$.
    \begin{align*}
        G_N(z) & = \sum_{k=0}^{N} (s_k - s_{k-1}) z^k \\
               & = \sum_{k=0}^{N} s_k z^k - \sum_{k=0}^{N} s_{k-1} z^k \\
               & = \sum_{k=0}^{N} s_k z^k - \sum_{j=-1}^{N-1} s_{j} z^{j+1} \quad (\text{substituting } j=k-1) \\
               & = \sum_{k=0}^{N} s_k z^k - z \sum_{j=0}^{N-1} s_{j} z^{j} \quad (\text{since } s_{-1}=0) \\
               & = (s_N z^N + \sum_{k=0}^{N-1} s_k z^k) - z \sum_{k=0}^{N-1} s_k z^k \\
               & = s_N z^N + (1-z) \sum_{k=0}^{N-1} s_k z^k.
    \end{align*}
    Since $|z| < 1$, the power series for $G(z)$ converges, which implies that $\lim_{k \to \infty} a_k z^k = 0$. This requires that the sequence $a_k$ is bounded, and therefore $|s_N| = |\sum_{k=0}^N a_k| \le \sum_{k=0}^N |a_k|$ grows at most linearly with $N$. Thus, as $N \to \infty$, the term $s_N z^N \to 0$ because $|z| < 1$.
    Taking the limit as $N \to \infty$, we get:
    \[
        G(z) = \lim_{N\to\infty} G_N(z) = (1-z) \sum_{k=0}^{\infty} s_k z^k.
    \]
    The series on the right converges because the original series for $G(z)$ converges.
\end{proof}

\begin{lemma}[Bound on the Tail Sum]
    \label{lem:Bound_on_the_Tail_Sum}
    \uses{def:Stolz_Sector}
    Let $(s_k)_{k=0}^{\infty}$ be a sequence of complex numbers such that $\lim_{k \to \infty} s_k = 0$. For any $\varepsilon > 0$, there exists an integer $N$ such that for all $n \geq N$ and for any $z$ within a Stolz sector defined by a constant $M > 1$, the following inequality holds:
    \[
        \left|(1-z)\sum_{k=n}^{\infty} s_k z^k\right| \leq M\varepsilon.
    \]
\end{lemma}
\begin{proof}
    Since $s_k \to 0$, for any given $\varepsilon > 0$, we can choose an integer $N$ such that for all $k \geq N$, we have $|s_k| < \varepsilon$.
    Let $n \geq N$. For any $z$ in the open unit disk, we can bound the tail of the series using the triangle inequality:
    \[
        \left|(1-z)\sum_{k=n}^{\infty} s_k z^k\right| \leq |1-z| \sum_{k=n}^{\infty} |s_k| |z|^k.
    \]
    Since $k \geq n \geq N$, we have $|s_k| < \varepsilon$. Therefore,
    \[
        |1-z| \sum_{k=n}^{\infty} |s_k| |z|^k < |1-z| \sum_{k=n}^{\infty} \varepsilon |z|^k = \varepsilon |1-z| \sum_{k=n}^{\infty} |z|^k.
    \]
    The sum is a geometric series: $\sum_{k=n}^{\infty} |z|^k = \frac{|z|^n}{1-|z|}$. So we have:
    \[
        \left|(1-z)\sum_{k=n}^{\infty} s_k z^k\right| < \varepsilon |1-z| \frac{|z|^n}{1-|z|}.
    \]
    Since $|z|^n < 1$ for $|z|<1$, we can simplify this to:
    \[
        \left|(1-z)\sum_{k=n}^{\infty} s_k z^k\right| < \varepsilon \frac{|1-z|}{1-|z|}.
    \]
    Now, if $z$ lies within a Stolz sector as defined in definition \ref{def:Stolz_Sector}, we have $|1-z| \leq M(1-|z|)$, which implies $\frac{|1-z|}{1-|z|} \leq M$.
    Substituting this into our inequality gives:
    \[
        \left|(1-z)\sum_{k=n}^{\infty} s_k z^k\right| \leq M\varepsilon.
    \]
\end{proof}

\begin{theorem}[Abel's Theorem]
    \label{thm:Abels_Theorem}
    \uses{def:Stolz_Sector}
    Let $G(z) = \sum_{k=0}^{\infty} a_k z^k$ be a power series with complex coefficients and radius of convergence $R=1$. If the series of coefficients $\sum_{k=0}^{\infty} a_k$ converges to a sum $S$, then
    \[
        \lim_{z \to 1} G(z) = S
    \]
    where the limit is taken as $z$ approaches $1$ from within any Stolz sector in the open unit disk.
\end{theorem}

\begin{proof}
    \uses{lem:Summation_by_Parts}
    \uses{lem:Bound_on_the_Tail_Sum}
    First, we consider the case where $S=0$. Let $s_n = \sum_{k=0}^{n} a_k$. Our assumption is that $\lim_{n \to \infty} s_n = 0$. We want to show that $\lim_{z \to 1} G(z) = 0$ as $z$ approaches $1$ within a Stolz sector.

    By lemma \ref{lem:Summation_by_Parts}, we can write $G(z) = (1-z) \sum_{k=0}^{\infty} s_k z^k$.
    Let $\varepsilon > 0$ be given. We want to show that $|G(z)|$ can be made arbitrarily small by choosing $z$ sufficiently close to 1 within a Stolz sector.

    We split the sum into two parts at an integer $n$ which we will choose later:
    \[
        G(z) = (1-z)\sum_{k=0}^{n-1} s_k z^k + (1-z)\sum_{k=n}^{\infty} s_k z^k.
    \]
    By the triangle inequality:
    \[
        |G(z)| \leq \left|(1-z)\sum_{k=0}^{n-1} s_k z^k\right| + \left|(1-z)\sum_{k=n}^{\infty} s_k z^k\right|.
    \]
    Since $s_k \to 0$, we can apply lemma \ref{lem:Bound_on_the_Tail_Sum}. For a given $\varepsilon' > 0$, we can choose $n$ large enough such that for $z$ in a Stolz sector defined by $M$, the second term is bounded:
    \[
        \left|(1-z)\sum_{k=n}^{\infty} s_k z^k\right| \leq M\varepsilon'.
    \]
    Now consider the first term. For this fixed $n$, the sum $\sum_{k=0}^{n-1} s_k z^k$ is a finite polynomial, and is therefore bounded in the unit disk. Let $K = \max_{0 \le k < n} |s_k|$. Then $|\sum_{k=0}^{n-1} s_k z^k| \leq \sum_{k=0}^{n-1} |s_k| \leq nK$. Thus,
    \[
        \left|(1-z)\sum_{k=0}^{n-1} s_k z^k\right| \leq |1-z| \cdot nK.
    \]
    As $z \to 1$, the factor $|1-z| \to 0$. So we can choose $\delta > 0$ such that if $|1-z| < \delta$, this term is less than $\varepsilon'$.

    Combining these results, for a given $\varepsilon > 0$, let $\varepsilon' = \varepsilon / (M+1)$. We can choose $n$ large enough so the tail is bounded by $M\varepsilon'$. Then, for that $n$, we can choose $\delta$ small enough so the head is bounded by $\varepsilon'$ for $|1-z| < \delta$.
    Thus, for $z$ in the Stolz sector and $|1-z| < \delta$, we have
    \[
        |G(z)| \leq \varepsilon' + M\varepsilon' = (M+1)\varepsilon' = \varepsilon.
    \]
    This shows that $\lim_{z \to 1} G(z) = 0$.

    For the general case where $\sum a_k = S \neq 0$, consider a new series $H(z) = \sum_{k=0}^\infty b_k z^k$, where $b_0 = a_0 - S$ and $b_k = a_k$ for $k \geq 1$. The sum of the coefficients of $H(z)$ is $\sum b_k = (a_0 - S) + \sum_{k=1}^\infty a_k = (\sum a_k) - S = S - S = 0$.
    From the case already proven, $\lim_{z \to 1} H(z) = 0$.
    But $G(z) = \sum a_k z^k = (a_0 - S)z^0 + \sum_{k=1}^\infty a_k z^k + S = H(z) + S$.
    Therefore,
    \[
        \lim_{z \to 1} G(z) = \lim_{z \to 1} (H(z) + S) = 0 + S = S.
    \]
    The theorem is proven.
\end{proof}
