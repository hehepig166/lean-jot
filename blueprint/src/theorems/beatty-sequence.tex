
\chapter{Rayleigh's Theorem on Beatty Sequences}

\begin{definition}[Beatty Sequence]
    \label{def:Beatty_Sequence}
    For a positive irrational number $r$, the Beatty sequence $\mathcal{B}_r$ is the set of integers given by $\mathcal{B}_r = \{\lfloor nr \rfloor \mid n \in \mathbb{Z}^+\}$, where $\lfloor \cdot \rfloor$ is the floor function.
\end{definition}

\begin{definition}[Complementary Beatty Sequences]
    \label{def:Complementary_Beatty_Sequences}
    \uses{def:Beatty_Sequence}
    Let $r > 1$ be an irrational number. Let $s$ be a real number such that $\frac{1}{r} + \frac{1}{s} = 1$. The sequences defined in definition \ref{def:Beatty_Sequence}, $\mathcal{B}_r$ and $\mathcal{B}_s$, are called a pair of complementary Beatty sequences. Note that since $r>1$ is irrational, $s = \frac{r}{r-1}$ must also be an irrational number greater than 1.
\end{definition}

\begin{lemma}[Disjointness of Complementary Beatty Sequences]
    \label{lem:Disjointness}
    \uses{def:Complementary_Beatty_Sequences}
    The two sequences in a pair of complementary Beatty sequences, $\mathcal{B}_r$ and $\mathcal{B}_s$, are disjoint, i.e., $\mathcal{B}_r \cap \mathcal{B}_s = \emptyset$.
\end{lemma}

\begin{proof}
    We prove this by contradiction. Assume there exists an integer $j$ that is in both sequences. Then, for some positive integers $k$ and $m$, we have $j = \lfloor kr \rfloor$ and $j = \lfloor ms \rfloor$.

    By the definition of the floor function, these equalities are equivalent to the inequalities:
    \[
    j \le kr < j+1 \quad \text{and} \quad j \le ms < j+1
    \]
    Since $r$ and $s$ are irrational, $kr$ and $ms$ cannot be integers. Thus, we can write the strict inequalities:
    \[
    j < kr < j+1 \quad \text{and} \quad j < ms < j+1
    \]
    Dividing by $r$ and $s$ respectively (which are positive), we get:
    \[
    \frac{j}{r} < k < \frac{j+1}{r} \quad \text{and} \quad \frac{j}{s} < m < \frac{j+1}{s}
    \]
    Adding these two inequalities gives:
    \[
    \frac{j}{r} + \frac{j}{s} < k+m < \frac{j+1}{r} + \frac{j+1}{s}
    \]
    Factoring out $j$ on the left and $j+1$ on the right yields:
    \[
    j\left(\frac{1}{r} + \frac{1}{s}\right) < k+m < (j+1)\left(\frac{1}{r} + \frac{1}{s}\right)
    \]
    From definition \ref{def:Complementary_Beatty_Sequences}, we know that $\frac{1}{r} + \frac{1}{s} = 1$. Substituting this into the inequality gives:
    \[
    j < k+m < j+1
    \]
    This is a contradiction, as there can be no integer $k+m$ strictly between two consecutive integers $j$ and $j+1$. Therefore, our initial assumption must be false, and the two sequences are disjoint.
\end{proof}

\begin{lemma}[Completeness of Complementary Beatty Sequences]
    \label{lem:Completeness}
    \uses{def:Complementary_Beatty_Sequences}
    The union of the two sequences in a pair of complementary Beatty sequences, $\mathcal{B}_r \cup \mathcal{B}_s$, contains every positive integer.
\end{lemma}

\begin{proof}
    We will show that for any positive integer $N$, the number of elements in $\mathcal{B}_r \cup \mathcal{B}_s$ that are less than or equal to $N$ is exactly $N$.

    Let $C_r(N)$ be the number of elements in $\mathcal{B}_r$ that are less than or equal to $N$.
    $C_r(N) = |\{k \in \mathbb{Z}^+ \mid \lfloor kr \rfloor \le N \}|$.
    The condition $\lfloor kr \rfloor \le N$ is equivalent to $kr < N+1$, since $kr$ is not an integer. This is equivalent to $k < \frac{N+1}{r}$.
    Since $k$ must be a positive integer, the number of such values of $k$ is $\lfloor \frac{N+1}{r} \rfloor$.
    Thus, $C_r(N) = \lfloor \frac{N+1}{r} \rfloor$.

    Similarly, the number of elements in $\mathcal{B}_s$ less than or equal to $N$, denoted $C_s(N)$, is $C_s(N) = \lfloor \frac{N+1}{s} \rfloor$.

    Since the sequences are disjoint from lemma \ref{lem:Disjointness}, the total number of elements in their union less than or equal to $N$ is $C(N) = C_r(N) + C_s(N) = \lfloor \frac{N+1}{r} \rfloor + \lfloor \frac{N+1}{s} \rfloor$.

    For any non-integer $x$, we have $x-1 < \lfloor x \rfloor < x$. Since $r$ and $s$ are irrational, $\frac{N+1}{r}$ and $\frac{N+1}{s}$ are not integers. Thus, we can write:
    \[
    \left(\frac{N+1}{r} - 1\right) + \left(\frac{N+1}{s} - 1\right) < C(N) < \frac{N+1}{r} + \frac{N+1}{s}
    \]
    Using the property $\frac{1}{r} + \frac{1}{s} = 1$ from definition \ref{def:Complementary_Beatty_Sequences}, this simplifies to:
    \[
    (N+1)\left(\frac{1}{r} + \frac{1}{s}\right) - 2 < C(N) < (N+1)\left(\frac{1}{r} + \frac{1}{s}\right)
    \]
    \[
    (N+1) - 2 < C(N) < N+1
    \]
    \[
    N-1 < C(N) < N+1
    \]
    Since $C(N)$ is an integer, the only possibility is $C(N) = N$.

    This means that for any integer $N$, the set $\{1, 2, \dots, N\}$ is composed of exactly the elements of $\mathcal{B}_r$ and $\mathcal{B}_s$ that are less than or equal to $N$. This holds for all $N$, so the union $\mathcal{B}_r \cup \mathcal{B}_s$ must be the set of all positive integers $\mathbb{Z}^+$.
\end{proof}

\begin{theorem}[Rayleigh's Theorem]
    \label{thm:Rayleighs_Theorem}
    \uses{def:Complementary_Beatty_Sequences}
    A pair of complementary Beatty sequences, $\mathcal{B}_r$ and $\mathcal{B}_s$, forms a partition of the set of positive integers $\mathbb{Z}^+$.
\end{theorem}

\begin{proof}
    \uses{lem:Disjointness}
    \uses{lem:Completeness}
    To show that the sequences $\mathcal{B}_r$ and $\mathcal{B}_s$ form a partition of $\mathbb{Z}^+$, we must show that they are disjoint and their union is $\mathbb{Z}^+$.

    Lemma \ref{lem:Disjointness} establishes that $\mathcal{B}_r \cap \mathcal{B}_s = \emptyset$.

    Lemma \ref{lem:Completeness} establishes that $\mathcal{B}_r \cup \mathcal{B}_s = \mathbb{Z}^+$.

    Therefore, the two sequences form a partition of the set of positive integers.
\end{proof}
