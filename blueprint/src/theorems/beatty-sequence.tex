\chapter{Rayleigh's Theorem}

\begin{definition}[Beatty Sequence]
For a positive irrational number $r > 1$, the Beatty sequence $\mathcal{B}_r$ is the set of integers given by $\mathcal{B}_r = \{ \lfloor nr \rfloor : n \in \mathbb{Z}^+ \}$.
\end{definition}

\begin{definition}[Partition of a Set]
A partition of a set $X$ is a collection of non-empty subsets of $X$ such that every element in $X$ is in exactly one of these subsets. For two subsets $A$ and $B$, this means $A \cap B = \emptyset$ and $A \cup B = X$.
\end{definition}

\begin{lemma}[Disjointness of Complementary Beatty Sequences]
Let $r > 1$ be an irrational number and let $s$ be a real number such that $\frac{1}{r} + \frac{1}{s} = 1$. The Beatty sequences $\mathcal{B}_r$ and $\mathcal{B}_s$ are disjoint, i.e., $\mathcal{B}_r \cap \mathcal{B}_s = \emptyset$.
\end{lemma}

\begin{proof}
Assume for the sake of contradiction that the sequences are not disjoint. Then there exists a positive integer $j$ such that $j \in \mathcal{B}_r$ and $j \in \mathcal{B}_s$.
By the definition of a Beatty sequence, this means there exist positive integers $k$ and $m$ such that:
$$j = \lfloor kr \rfloor \quad \text{and} \quad j = \lfloor ms \rfloor$$
These equalities can be rewritten as the inequalities:
$$j \leq kr < j+1 \quad \text{and} \quad j \leq ms < j+1$$
Since $r$ is irrational, $s = r/(r-1)$ must also be irrational. If $s$ were rational, $s = p/q$, then $r = s/(s-1) = (p/q)/((p/q)-1) = p/(p-q)$, which would imply $r$ is rational, a contradiction. Therefore, as $r$ and $s$ are irrational, $kr$ and $ms$ cannot be integers. The inequalities must be strict:
$$j < kr < j+1 \quad \text{and} \quad j < ms < j+1$$
Dividing by $r$ and $s$ respectively gives:
$$\frac{j}{r} < k < \frac{j+1}{r} \quad \text{and} \quad \frac{j}{s} < m < \frac{j+1}{s}$$
Adding these two inequalities, we obtain:
$$\frac{j}{r} + \frac{j}{s} < k+m < \frac{j+1}{r} + \frac{j+1}{s}$$
Using the given condition $\frac{1}{r} + \frac{1}{s} = 1$, this simplifies to:
$$j\left(\frac{1}{r} + \frac{1}{s}\right) < k+m < (j+1)\left(\frac{1}{r} + \frac{1}{s}\right)$$
$$j < k+m < j+1$$
This result is a contradiction, as there can be no integer $k+m$ strictly between two consecutive integers $j$ and $j+1$. Thus, our initial assumption must be false, and the sequences $\mathcal{B}_r$ and $\mathcal{B}_s$ are disjoint.
\end{proof}

\begin{lemma}[Completeness of Complementary Beatty Sequences]
Let $r > 1$ be an irrational number and let $s$ be a real number such that $\frac{1}{r} + \frac{1}{s} = 1$. The union of the Beatty sequences $\mathcal{B}_r$ and $\mathcal{B}_s$ covers all positive integers, i.e., $\mathcal{B}_r \cup \mathcal{B}_s = \mathbb{Z}^+$.
\end{lemma}

\begin{proof}
Assume for the sake of contradiction that there exists a positive integer $j$ that is in neither sequence. This means that for any positive integer $n$, $j \neq \lfloor nr \rfloor$ and $j \neq \lfloor ns \rfloor$.
This implies that there must exist non-negative integers $k$ and $m$ such that no multiple of $r$ or $s$ falls in the interval $[j, j+1)$.
This can be expressed by the following inequalities:
$$kr < j \quad \text{and} \quad j+1 \leq (k+1)r$$
$$ms < j \quad \text{and} \quad j+1 \leq (m+1)s$$
As $r$ and $s$ are irrational, the equalities can be excluded:
$$j+1 < (k+1)r \quad \text{and} \quad j+1 < (m+1)s$$
From these inequalities, we can deduce two sets of relations. First, from $kr < j$ and $ms < j$:
$$k < \frac{j}{r} \quad \text{and} \quad m < \frac{j}{s}$$
Adding these gives:
$$k+m < \frac{j}{r} + \frac{j}{s} = j\left(\frac{1}{r} + \frac{1}{s}\right) = j$$
So, $k+m \le j-1$.
Second, from $j+1 < (k+1)r$ and $j+1 < (m+1)s$:
$$\frac{j+1}{r} < k+1 \quad \text{and} \quad \frac{j+1}{s} < m+1$$
Adding these gives:
$$\frac{j+1}{r} + \frac{j+1}{s} < (k+1) + (m+1)$$
$$(j+1)\left(\frac{1}{r} + \frac{1}{s}\right) < k+m+2$$
$$j+1 < k+m+2 \implies j-1 < k+m$$
So, $k+m \ge j$.
We have derived two contradictory conditions: $k+m \le j-1$ and $k+m \ge j$. This is impossible.
Therefore, our assumption that such an integer $j$ exists is false, and every positive integer must belong to at least one of the sequences.
\end{proof}

\begin{theorem}[Rayleigh's Theorem]
Let $r > 1$ be an irrational number and let $s = r/(r-1)$. Then $s$ is also an irrational number greater than 1, they satisfy $\frac{1}{r} + \frac{1}{s} = 1$, and the two Beatty sequences $\mathcal{B}_r$ and $\mathcal{B}_s$ form a partition of the set of positive integers $\mathbb{Z}^+$.
\end{theorem}

\begin{proof}
The condition $s = r/(r-1)$ directly implies $\frac{1}{r} + \frac{1}{s} = 1$. Since $r > 1$, we have $r-1 > 0$, so $s > 0$. Also, since $r > r-1$, it follows that $s > 1$. As shown in the proof of Lemma 1, if $r$ is irrational, then $s$ must also be irrational.
Lemma 1 establishes that the sequences $\mathcal{B}_r$ and $\mathcal{B}_s$ are disjoint.
Lemma 2 establishes that their union covers the entirety of $\mathbb{Z}^+$.
Taken together, these two conditions mean that $\mathcal{B}_r$ and $\mathcal{B}_s$ form a partition of the set of positive integers.
\end{proof}
