\chapter{The Brunn-Minkowski Theorem}

\begin{definition}[Minkowski Sum]
    \label{def:Minkowski_Sum}
    Let $A$ and $B$ be two subsets of $\mathbb{R}^n$. The Minkowski sum of $A$ and $B$, denoted as $A+B$, is the set formed by the vector addition of elements from $A$ and $B$:
    \[ A+B := \{ a+b \in \mathbb{R}^n \mid a \in A, b \in B \}. \]
\end{definition}

\begin{definition}[Lebesgue Measure]
    \label{def:Lebesgue_Measure}
    Let $\mu$ denote the Lebesgue measure on $\mathbb{R}^n$. For a measurable set $S \subseteq \mathbb{R}^n$, $\mu(S)$ represents the $n$-dimensional volume of $S$. The measure is translation-invariant, meaning $\mu(S+x) = \mu(S)$ for any $x \in \mathbb{R}^n$. For a scalar $\lambda \in \mathbb{R}$, $\mu(\lambda S) = |\lambda|^n \mu(S)$.
\end{definition}

\begin{lemma}[Inequality for Axis-Aligned Boxes]
    \label{lem:Axis_Aligned_Boxes}
    \uses{def:Minkowski_Sum}
    \uses{def:Lebesgue_Measure}
    Let $A = \prod_{i=1}^{n}[0, a_i]$ and $B = \prod_{i=1}^{n}[0, b_i]$ be two axis-aligned boxes in $\mathbb{R}^n$ with $a_i, b_i \geq 0$. Then the Brunn-Minkowski inequality holds:
    \[ [\mu(A+B)]^{1/n} \geq [\mu(A)]^{1/n} + [\mu(B)]^{1/n}. \]
\end{lemma}

\begin{proof}
    The Minkowski sum of the two boxes is $A+B = \prod_{i=1}^{n}[0, a_i+b_i]$. The volumes are $\mu(A) = \prod_{i=1}^{n} a_i$, $\mu(B) = \prod_{i=1}^{n} b_i$, and $\mu(A+B) = \prod_{i=1}^{n} (a_i+b_i)$. The inequality can be written as:
    \[ \left(\prod_{i=1}^{n} (a_i+b_i)\right)^{1/n} \geq \left(\prod_{i=1}^{n} a_i\right)^{1/n} + \left(\prod_{i=1}^{n} b_i\right)^{1/n}. \]
    If either $\mu(A)=0$ or $\mu(B)=0$, the inequality is trivial. Assuming both are positive, we can divide by the left-hand side to get:
    \[ 1 \geq \left(\prod_{i=1}^{n} \frac{a_i}{a_i+b_i}\right)^{1/n} + \left(\prod_{i=1}^{n} \frac{b_i}{a_i+b_i}\right)^{1/n}. \]
    By the inequality of arithmetic and geometric means (AM-GM), we have:
    \[ \left(\prod_{i=1}^{n} \frac{a_i}{a_i+b_i}\right)^{1/n} \leq \frac{1}{n}\sum_{i=1}^{n} \frac{a_i}{a_i+b_i}, \]
    \[ \left(\prod_{i=1}^{n} \frac{b_i}{a_i+b_i}\right)^{1/n} \leq \frac{1}{n}\sum_{i=1}^{n} \frac{b_i}{a_i+b_i}. \]
    Summing these two inequalities gives:
    \[ \left(\prod_{i=1}^{n} \frac{a_i}{a_i+b_i}\right)^{1/n} + \left(\prod_{i=1}^{n} \frac{b_i}{a_i+b_i}\right)^{1/n} \leq \frac{1}{n}\sum_{i=1}^{n} \frac{a_i+b_i}{a_i+b_i} = \frac{1}{n}\sum_{i=1}^{n} 1 = 1. \]
    This establishes the desired result for axis-aligned boxes.
\end{proof}

\begin{lemma}[Inequality for Finite Unions of Disjoint Boxes]
    \label{lem:Finite_Union_of_Disjoint_Boxes}
    \uses{lem:Axis_Aligned_Boxes}
    Let $A$ and $B$ be finite unions of disjoint, axis-aligned boxes. The Brunn-Minkowski inequality holds for $A$ and $B$.
\end{lemma}

\begin{proof}
    We proceed by induction on the total number of boxes in $A \cup B$. The base case, where $A$ and $B$ are single boxes, is proven in Lemma \ref{lem:Axis_Aligned_Boxes}.

    For the inductive step, assume the inequality holds for any pair of sets with a total of fewer than $m$ boxes. Let $A$ and $B$ be composed of $m$ boxes in total. We can choose an axis-aligned hyperplane $H$ that splits one of the sets, say $A$, into two non-empty subsets $A_1$ and $A_2$, which are unions of boxes lying on opposite sides of $H$. Let $H$ be defined by $x_1 = c$. We define $A_1 = A \cap \{x_1 < c\}$ and $A_2 = A \cap \{x_1 > c\}$.

    We can translate $B$ along the $x_1$-axis by some amount $t$ to form $B_t=B+(t,0,...,0)$. The ratio $\mu(B_t \cap \{x_1 < c\}) / \mu(B_t \cap \{x_1 > c\})$ varies continuously from $\infty$ to $0$ as $t$ goes from $-\infty$ to $\infty$. By the Intermediate Value Theorem, there exists a translation $B'$ of $B$ such that
    \[ \frac{\mu(A_1)}{\mu(B'_1)} = \frac{\mu(A_2)}{\mu(B'_2)} = \frac{\mu(A)}{\mu(B)}, \]
    where $B'_1$ and $B'_2$ are the parts of $B'$ on each side of $H$.

    Note that $A_1+B'_1$ and $A_2+B'_2$ are disjoint. Thus, $\mu(A+B') \geq \mu(A_1+B'_1) + \mu(A_2+B'_2)$.
    By the induction hypothesis applied to the pairs $(A_1, B'_1)$ and $(A_2, B'_2)$,
    \begin{align*}
        \mu(A+B) = \mu(A+B') & \geq [\mu(A_1)^{1/n} + \mu(B'_1)^{1/n}]^n + [\mu(A_2)^{1/n} + \mu(B'_2)^{1/n}]^n \\
        & = \left(1 + \frac{\mu(A_1)^{1/n}}{\mu(B'_1)^{1/n}}\right)^n \mu(B'_1) + \left(1 + \frac{\mu(A_2)^{1/n}}{\mu(B'_2)^{1/n}}\right)^n \mu(B'_2).
    \end{align*}
    Using the ratio property $\frac{\mu(A_i)^{1/n}}{\mu(B'_i)^{1/n}} = \frac{\mu(A)^{1/n}}{\mu(B)^{1/n}}$, the expression simplifies to:
    \begin{align*}
        & \left(1 + \frac{\mu(A)^{1/n}}{\mu(B)^{1/n}}\right)^n (\mu(B'_1) + \mu(B'_2)) \\
        = & \left(\frac{\mu(B)^{1/n} + \mu(A)^{1/n}}{\mu(B)^{1/n}}\right)^n \mu(B) \\
        = & ([\mu(A)]^{1/n} + [\mu(B)]^{1/n})^n.
    \end{align*}
    Taking the $n$-th root of both sides completes the induction.
\end{proof}

\begin{lemma}[Inequality for Bounded Open Sets]
    \label{lem:Bounded_Open_Sets}
    \uses{lem:Finite_Union_of_Disjoint_Boxes}
    The Brunn-Minkowski inequality holds for any two non-empty, bounded, open sets $A, B \subset \mathbb{R}^n$.
\end{lemma}

\begin{proof}
    Any bounded open set in $\mathbb{R}^n$ can be expressed as the union of a countable collection of disjoint axis-aligned boxes. Thus, for any $\epsilon > 0$, we can find finite unions of disjoint boxes, $A_k \subset A$ and $B_k \subset B$, such that $\mu(A) - \mu(A_k) < \epsilon$ and $\mu(B) - \mu(B_k) < \epsilon$.
    The Minkowski sum $A_k + B_k$ is a subset of $A+B$. Applying Lemma \ref{lem:Finite_Union_of_Disjoint_Boxes} to $A_k$ and $B_k$:
    \[ [\mu(A_k+B_k)]^{1/n} \geq [\mu(A_k)]^{1/n} + [\mu(B_k)]^{1/n}. \]
    Since $A_k+B_k \subseteq A+B$, we have $\mu(A+B) \geq \mu(A_k+B_k)$. Therefore,
    \[ [\mu(A+B)]^{1/n} \geq [\mu(A_k)]^{1/n} + [\mu(B_k)]^{1/n}. \]
    Taking the limit as $\epsilon \to 0$, we have $\mu(A_k) \to \mu(A)$ and $\mu(B_k) \to \mu(B)$. The inequality holds in the limit, proving the statement for bounded open sets.
\end{proof}

\begin{lemma}[Inequality for Compact Sets]
    \label{lem:Compact_Sets}
    \uses{lem:Bounded_Open_Sets}
    The Brunn-Minkowski inequality holds for any two non-empty, compact sets $A, B \subset \mathbb{R}^n$.
\end{lemma}

\begin{proof}
    For any $\epsilon > 0$, define the $\epsilon$-thickening of a set $X$ as $X_\epsilon = X + B(0, \epsilon)$, where $B(0, \epsilon)$ is the open ball of radius $\epsilon$ centered at the origin. If $A$ and $B$ are compact, then $A_\epsilon$ and $B_\epsilon$ are bounded open sets. From the properties of the Minkowski sum, we have $A_\epsilon + B_\epsilon = (A+B)_{2\epsilon}$.

    Applying Lemma \ref{lem:Bounded_Open_Sets} to $A_\epsilon$ and $B_\epsilon$:
    \[ [\mu(A_\epsilon+B_\epsilon)]^{1/n} \geq [\mu(A_\epsilon)]^{1/n} + [\mu(B_\epsilon)]^{1/n}. \]
    Substituting $A_\epsilon + B_\epsilon = (A+B)_{2\epsilon}$, we get:
    \[ [\mu((A+B)_{2\epsilon})]^{1/n} \geq [\mu(A_\epsilon)]^{1/n} + [\mu(B_\epsilon)]^{1/n}. \]
    As $\epsilon \to 0$, the volume of the $\epsilon$-thickening of a compact set converges to the volume of the set itself. That is, $\lim_{\epsilon\to 0} \mu(X_\epsilon) = \mu(X)$. Taking the limit as $\epsilon \to 0$ on both sides of the inequality, we obtain:
    \[ [\mu(A+B)]^{1/n} \geq [\mu(A)]^{1/n} + [\mu(B)]^{1/n}. \]
\end{proof}

\begin{theorem}[The Brunn-Minkowski Theorem]
    \label{thm:Brunn-Minkowski_Theorem}
    \uses{def:Minkowski_Sum}
    \uses{def:Lebesgue_Measure}
    \uses{lem:Compact_Sets}
    Let $A$ and $B$ be two non-empty, measurable subsets of $\mathbb{R}^n$ such that $A+B$ is also measurable. Then the following inequality holds:
    \[ [\mu(A+B)]^{1/n} \geq [\mu(A)]^{1/n} + [\mu(B)]^{1/n}. \]
\end{theorem}

\begin{proof}
    The proof extends the result from compact sets to general measurable sets via an approximation argument.

    First, assume $A$ and $B$ are bounded. By the regularity of the Lebesgue measure, for any $\delta > 0$, there exist compact sets $A_k \subset A$ and $B_k \subset B$ such that $\mu(A \setminus A_k) < \delta$ and $\mu(B \setminus B_k) < \delta$. We have $A_k+B_k \subset A+B$, so $\mu(A+B) \geq \mu(A_k+B_k)$.
    Applying Lemma \ref{lem:Compact_Sets} to $A_k$ and $B_k$:
    \[ [\mu(A+B)]^{1/n} \geq [\mu(A_k+B_k)]^{1/n} \geq [\mu(A_k)]^{1/n} + [\mu(B_k)]^{1/n}. \]
    As we let $\delta \to 0$, we have $\mu(A_k) \to \mu(A)$ and $\mu(B_k) \to \mu(B)$. The inequality thus holds for bounded measurable sets.

    Now, let $A$ and $B$ be unbounded. For any $k \in \mathbb{N}$, define $A_k = A \cap [-k,k]^n$ and $B_k = B \cap [-k,k]^n$. These sets are bounded and measurable. From the result for bounded sets:
    \[ [\mu(A_k+B_k)]^{1/n} \geq [\mu(A_k)]^{1/n} + [\mu(B_k)]^{1/n}. \]
    Since $A_k+B_k \subset A+B$, we have $\mu(A+B) \geq \mu(A_k+B_k)$. As $k \to \infty$, $A_k$ and $B_k$ exhaust $A$ and $B$, so $\mu(A_k) \to \mu(A)$ and $\mu(B_k) \to \mu(B)$ by the monotone convergence theorem for measures. Taking the limit as $k \to \infty$ establishes the theorem for all non-empty measurable sets.
\end{proof}
